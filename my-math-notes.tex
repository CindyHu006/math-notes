\documentclass[11pt, oneside]{article}   	% use "amsart" instead of "article" for AMSLaTeX format
%\usepackage{geometry}                		% See geometry.pdf to learn the layout options. There are lots.
%\geometry{letterpaper}                   		% ... or a4paper or a5paper or ... 
%\geometry{landscape}                		% Activate for rotated page geometry
%\usepackage[parfill]{parskip}    		% Activate to begin paragraphs with an empty line rather than an indent

\usepackage{geometry}
 \geometry{
 a4paper,
 total={170mm,257mm},
 left=20mm,
 top=15mm,
 }

\usepackage{graphicx}				% Use pdf, png, jpg, or eps§ with pdflatex; use eps in DVI mode
								% TeX will automatically convert eps --> pdf in pdflatex		
\usepackage{amssymb}
\usepackage{amsmath}
\usepackage{fancyhdr}
\usepackage[utf8]{inputenc}
\usepackage[english]{babel}
\usepackage{enumerate}
\usepackage{arcs}
\usepackage{cancel}
\usepackage{xfrac}

%SetFonts

%SetFonts

\usepackage[inline]{asymptote}


\pagestyle{fancy}
\fancyhf{}
%\rhead{Teacher David @ 18601688612}
\lhead{\leftmark}


\title{Lecture Notes}
\author{Cindy Hu}
%\date{}							% Activate to display a given date or no date

\begin{document}
\maketitle




\section{Quadratic Equations}
Given quadratic equation $ax^2+bx+c=0, a \ne 0, a, b, c \in \mathbb{R}$, we can solve it by the technique of \emph{Completing Square}: 
\begin{align*}
& ax^2+bx+c=0\\
\Rightarrow \quad & x^2+\frac{b}{a}x+\frac{c}{a}=0\\
\Rightarrow \quad & (x+\frac{b}{2a})^2-\frac{b^2}{4a^2}+\frac{c}{a}=0\\
%\Rightarrow \quad & (x+\frac{b}{2a})^2=\frac{b^2}{4a^2}-\frac{c}{a}\\
\Rightarrow \quad & (x+\frac{b}{2a})^2=\frac{b^2-4ac}{4a^2}\\
\Rightarrow \quad & x+\frac{b}{2a}=\pm \frac{\sqrt{b^2-4ac}}{2a}\\
\Rightarrow \quad & x=\frac{-b \pm \sqrt{b^2-4ac}}{2a}
\end{align*}


%\renewcommand{\labelenumii}{(\arabic{enumii})}


%\begin{enumerate}
%\renewcommand{\labelenumi}{1.\arabic{enumi}}

%\item \label{rule:add} \emph{Addition Principle}. If there are $a$ varieties of soup and $b$ varieties of salad, then there are $a+b$ possible ways to order a meal of soup \emph{or} salad (but not both soup and salad). 

%\end{enumerate}




\end{document} 